\documentclass[12pt,a4j]{jsarticle}
\usepackage[%
top    = 30truemm,%
bottom = 30truemm,%
left   = 25truemm,%
right  = 25truemm]{geometry}
\usepackage{amsmath,amssymb,amsthm}
\usepackage{bm,braket,ascmac}


\newcommand{\p}{\partial}

\def\labelenumi{(\arabic{enumi})}

\title{B\'{e}nard 対流問題}
\author{@mat\_der\_D}
\date{\today}

\begin{document}
\maketitle


\section{基礎方程式}

基礎方程式は, Boussinesq モデルを用いる:
\begin{equation}
 \dfrac{\p \bm{u}}{\p t}
  + (\bm{u} \cdot \nabla) \bm{u}
  = - \nabla p + Pr Ra T \bm{k}
    + Pr \nabla^{2} \bm{u},
\end{equation}
\begin{equation}
 \nabla \cdot \bm{u} = 0,
\end{equation}
\begin{equation}
 \dfrac{\p T}{\p t} 
  + (\bm{u} \cdot \nabla) T
  = \nabla^{2} T.
\end{equation}
ここで $\bm{u} = (u, v, w)$ は (無次元) 速度場,
$T$ は (無次元) 温度場
\footnote{正確にはある基準温度からの温度差
(を無次元化したもの). 方程式は一見
温度の基準の取り方によって変わるように見えるが,
基準の取り換えによる項は圧力項に吸収できるため,
そのような心配はない.},
$p$ は (無次元) 圧力場であり,
$\bm{k}$ は $z$ 方向の単位ベクトルである.
無次元化に用いた長さスケールを $L$,
時間スケールを $\tau$, 速度スケールを $U$,
温度スケールを $\Theta$ とする.
これらのスケールの間には
\begin{equation}
  \tau = \dfrac{L^2}{\kappa}, \quad
  U    = \dfrac{L}{\tau} = \dfrac{\kappa}{L}
  \notag
\end{equation}
が成り立つ. ただし $\kappa$ は温度拡散率である.
また無次元パラメーターである
Prandtl 数 $Pr$, Rayleigh 数 $Ra$ は
\begin{equation}
 Pr = \dfrac{\nu}{\kappa}, \quad
 Ra = \dfrac{\alpha g \Theta L^{3}}{\kappa \nu}
 \notag
\end{equation}
で定義する. \par

  二次元の場合 (全ての変数が $y$ に依存せず, $v \equiv 0$ の場合)
について考える. このときは変数はすべて $x, z$ の関数であり,
$\bm{u} = (u, w)$, $\nabla = (\frac{\p}{\p x}, \frac{\p}{\p z})$
と解釈することとする. 流れ関数 $\psi(x, z)$ を
\begin{equation}
 u =   \dfrac{\p \psi}{\p z}, \quad
 w = - \dfrac{\p \psi}{\p x}
 \notag
\end{equation}
と導入することで, 基礎方程式は次のように変形される
\footnote{非圧縮条件 $\nabla \cdot \bm{u} = 0$ は
流れ関数を導入した時点で自動的に満足する.}:
\begin{equation}
 \dfrac{\p \omega}{\p t}
  + J(\psi, \omega)
  = - Pr Ra \dfrac{\p T}{\p x}
    + Pr \nabla^{2} \omega,
\end{equation}
\begin{equation}
 \dfrac{\p T}{\p t} + J(\psi, T) = \nabla^{2} T.
\end{equation}
ここで $\omega = - \nabla^{2} \psi$ は $y$ 方向の
渦度である. また $a(x, z), b(x, z)$ に対して
\begin{equation}
 J(a, b) = 
  \dfrac{\p a}{\p z} \dfrac{\p b}{\p x}
  -
  \dfrac{\p a}{\p x} \dfrac{\p b}{\p z}
\end{equation}
と定義する.

\section{境界条件}

速度の境界条件は, 運動学的条件 $+$ 応力なし条件:
       \begin{equation}
	\dfrac{\p u}{\p z}
	 = \dfrac{\p v}{\p z}
	 = w
	 = 0
	 \qquad (\text{at } z = \text{境界})
       \end{equation}
       を課す.
       2次元の場合, 流れ関数を用いて書き換えると
       \begin{equation}
	\psi = \dfrac{\p^{2} \psi}{\p z^{2}}
	 = 0
	 \qquad (\text{at } z = \text{境界})
       \end{equation}
       となる\footnote{厳密には, 水平方向の流量が
       常に $0$ である条件も課している.}.
\end{enumerate}
温度の境界条件は
\begin{equation}
 T = T_{L} \qquad (\text{at } z = \text{下の境界}),
  \qquad
 T = T_{U} \qquad (\text{at } z = \text{上の境界})
\end{equation}
とする. ただし $T_{L} > T_{U}$. \par

ここで無次元化に用いた長さスケール $L$ を
上下境界の間の距離とし, 温度スケール $\Theta$ を
$T_{L} - T_{U}$ と定めれば,
境界を $z = 0, 1$ とでき, また温度に関する境界条件を
$T(z = 0) = 1, T(z = 1) = 0$ とできる.

\end{document}