\documentclass[12pt,a4j]{jsarticle}
\usepackage[%
top    = 30truemm,%
bottom = 30truemm,%
left   = 25truemm,%
right  = 25truemm]{geometry}
\usepackage{amsmath,amssymb,amsthm}
\usepackage{bm,braket,ascmac}


% (RE)NEW COMMANDS
\newcommand{\p}{\partial}
\newcommand{\R}{\mathbb{R}}

\begin{document}


非圧縮の $2$ 次元 Navier-Stokes 方程式を考える:
\begin{equation}
 u_{t} + u u_{x} + v u_{y} 
  = - p_{x} + Re^{-1} (u_{xx} + u_{yy}),
  \label{NSeq x}
\end{equation}
\begin{equation}
 v_{t} + u v_{x} + v v_{y}
  = - p_{y} + Re^{-1} (v_{xx} + v_{yy}),
  \label{NSeq y}
\end{equation}
\begin{equation}
 u_{x} + v_{y} = 0.
  \label{incomp}
\end{equation}
流れ関数 $\psi$ を
\begin{equation}
  u =   \psi_{y}, \qquad
  v = - \psi_{x}
  \label{stream func}
\end{equation}
で導入すると, 非圧縮条件 (\ref{incomp}) は自動的に満たされる.
渦度 $\omega$ を
\begin{equation}
 \omega = - u_{y} + v_{x} = - (\psi_{xx} + \psi_{yy})
  \label{vorticity}
\end{equation}
で導入すると, (\ref{NSeq x}), (\ref{NSeq y}) から
\begin{equation}
 \omega_{t} + J(\psi, \omega) 
  = Re^{-1} (\omega_{xx} + \omega_{yy})
  \label{vorticity eq}
\end{equation}
が得られる. ここで
\begin{equation}
 J(\psi, \omega) = \psi_{y} \omega_{x} - \psi_{x} \omega_{y}
  \label{def Jacobi}
\end{equation}
である. \par

流体の領域は
\begin{equation}
 [- L_{x}, L_{x}] \times [- L_{y}, L_{y}]
  \setminus
  \left[ -\dfrac{1}{2}, \dfrac{1}{2} \right]
  \times
  \left[ -\dfrac{1}{2}, \dfrac{1}{2} \right]
  \subseteq
  \R^{2}
\end{equation}
とする. 境界条件は,
$y = \pm L_{y}$ では周期境界条件,
$x = \pm 1/2$ および $y = \pm 1/2$ の境界では
粘着境界条件
\begin{equation}
 u = v = 0,
\end{equation}
$x = - L_{x}$ では固定速度の境界条件
\begin{equation}
 u = 1, \quad
 v = 0,
\end{equation}
$x = L_{y}$ ではノイマン条件
\begin{equation}
 u_{x} = v_{x} = 0
\end{equation}
を課す.

\end{document}