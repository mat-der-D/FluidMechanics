\documentclass[12pt, a4j]{jsarticle}
\usepackage[%
top    = 30truemm,%
bottom = 30truemm,%
left   = 25truemm,%
right  = 25truemm]{geometry}
\usepackage{amsmath,amssymb,amsthm}
\usepackage{bm,braket,ascmac}


%(RE)NEW COMMANDS
\newcommand{\R}{\mathbb{R}}
\newcommand{\p}{\partial}


\begin{document}

次の方程式を解く:
\begin{equation}
 u_{t} + 6 u u_{x} + u_{xxx} = 0.
  \label{KdV}
\end{equation}
ただし 
$u:\R \times [0, \infty) \ni (x, t) \to u(x, t) \in \R$
である. ここでは周期境界条件
\begin{equation}
 u(x, t) = u(x + L, t)
  \quad (\forall (x, t) \in \R \times [0, \infty) )
  \notag
\end{equation}
を満たすものを考える. ある正の整数 $N$ に対して
\begin{equation}
 x_{i} = \dfrac{i}{N} \quad (0 \leq i \leq N)
  \notag
\end{equation}
と定め,
\begin{equation}
 0 = x_{0} < x_{1} < \dots < x_{N-1} < x_{N} = L
  \notag
\end{equation}
と差分化する. 今 $\Delta x = L/N$ とおくと
\begin{align}
 u_{x}(x_{i}, t)
 &= \dfrac{u(x_{i+1}, t) - u(x_{i-1}, t)}{2 \Delta x}
    + O((\Delta x)^{2}), \notag \\
 u_{xxx}(x_{i}, t)
 &= \dfrac{
      u(x_{i+2}, t)
      - 2*u(x_{i+1}, t)
      + 2*u(x_{i-1}, t)
      - u(x_{i-2}, t)
    }{2 (\Delta x)^{3}}
    + O((\Delta x)^{2})
 \notag
\end{align}
となる. よって $O((\Delta x)^{2})$ の精度で
\begin{equation}
 u_t = F(u),
  \notag
\end{equation}
\begin{align}
 F(u)
  &= \left(
     - 6 u(x_{i}, t) \cdot
     \dfrac{u(x_{i+1}, t) - u(x_{i-1}, t)}{2 \Delta x}
     \right.
  \notag \\
  & \quad \quad \quad \left.
     \dfrac{
      - u(x_{i+2}, t)
      - 2*u(x_{i+1}, t)
      + 2*u(x_{i-1}, t)
      - u(x_{i-2}, t)
    }{2 (\Delta x)^{3}}
    \right)_{0 \leq i \leq N}
  \notag
\end{align}
となる. ここではこれを利用して4次の Runge-Kutta 法により
時間積分を行う.

\end{document}